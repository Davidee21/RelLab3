\documentclass{article}
\usepackage[italian]{babel}
\usepackage[utf8]{inputenc}
\usepackage{fancyhdr}
\usepackage{tikz}
\usepackage{amsmath}
\usepackage{amssymb}
\usepackage{amsthm}
\usepackage{amsfonts}
\usepackage{color}
\usepackage{circuitikz}
\usepackage[margin=2cm]{geometry}
\usepackage[scientific-notation=true]{siunitx}
\usepackage{titlesec}
\usepackage{graphics}

\titleformat{\paragraph}
  {\normalfont\normalsize\bfseries}{\theparagraph}{1em}{}
\titlespacing*{\paragraph}
  {0pt}{3.25ex plus 1ex minus .2ex}{1.5ex plus .2ex}

\title{Misura della caratteristica di due diodi a giunzione p-n}
\date{Quarto turno}
\author{Bertasi Leonardo, Perniola Davide }
\begin{document}
\maketitle
\section{Introduzione} 
Lo scopo della prova è stato quello di misurare la caratteristica I-V di due diodi a semiconduttore, in particolare uno al silicio e uno al germanio...qualcosa id più sugli obb
L'apparato sperimentale era composto un alimentatore di bassa tensione, un multimetro digitale, un oscilloscopio, un potenziometro da $1k\Omega$ oltre che dai due diodi in esame.
Il circuito realizzato è riporato in Figura 1.


\section{Risultati}


\section{Conclusioni}











\end{document}
